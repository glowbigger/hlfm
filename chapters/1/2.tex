\section{Truth functions and truth tables}

\note{} For readability of truth tables, the value 1 will be used for \(T\) and the value 0 will be used for \(F\). The final operation to be evaluated, which indicates the truth value of the final statement form, will be underlined.

\solutions{}
\begin{enumerate}
  \setcounter{enumi}{2}
  \item % 3
    Because truth tables are tedious to write, only (a) has been done. The rest are done similarly.
    \begin{enumerate}[(\alph*), align=left]
      \item Observe the truth table below.
        \begin{center}
          \begin{tabular}{ccccc}
            \(((\sim\)&
            \(p)\)&
            \(\underline{\wedge}\)&
            \((\sim\)&
            \(q))\)\\

            0&
            1&
            0&
            0&
            1\\

            0&
            1&
            0&
            1&
            0\\

            1&
            0&
            0&
            0&
            1\\

            1&
            0&
            1&
            1&
            0
          \end{tabular}
        \end{center}
    \end{enumerate}

  \item % 4
    When \(p\) and \(q\) take on particular values, the values of \(((\sim p) \lor q)\) and \((p \ra q)\) are identical. This can be shown by constructing truth tables, but that process is omitted here. Similarly, \(((\sim p) \ra (q \lor r))\) can be shown to give rise to the same truth function as \(((\sim q) \ra ((\sim r) \ra p))\) by constructing truth tables.

  \item % 5
    The statement forms (a), (b), and (d) are tautologies.

  \item % 6
    Because truth tables are tedious to write, only (a) has been done. The rest are done similarly.
    \begin{enumerate}[(\alph*), align=left]
      \item The truth table for \((p \ra q)\) is seen below.

        \begin{center}
          \begin{tabular}{ccc}
            \((p\)&
            \(\underline{\ra}\)&
            \(q)\)\\

            1&
            1&
            1\\

            1&
            0&
            0\\

            0&
            1&
            1\\

            0&
            1&
            0
          \end{tabular}
        \end{center}

        The truth table for \(((\sim q) \ra (\sim p))\) is seen below.

        \begin{center}
          \begin{tabular}{ccccc}
            \(((\sim\)&
            \(q)\)&
            \(\underline{\ra}\)&
            \((\sim\)&
            \(p))\)\\

            0&
            1&
            1&
            0&
            1\\

            0&
            1&
            1&
            1&
            0\\

            1&
            0&
            0&
            0&
            1\\

            1&
            0&
            1&
            1&
            0
          \end{tabular}
        \end{center}

        Notice that when \(p\) and \(q\) take on the same values in both truth tables, the values underneath the underlined \(\ra\), which indicate the final operation to be evaluated, and therefore the truth value of the statement form, are identical. Therefore, the two statement forms are logically equivalent.
    \end{enumerate}

  \item % 7
    When \(p\) and \(q\) are both true, the value \((((\sim p) \ra q) \ra (p \ra (\sim q)))\) is false, so the statement form is not a tautology.

    Let \(\cua = \cub = (p \ra p)\). Notice that this is a tautology, and for that matter, any tautology can be substituted here. The value of \((((\sim \cua) \ra \cub) \ra (\cua \ra (\sim \cub)))\) can then be seen to be a contradiction by constructing the appropriate truth table or in the same way that the above statement form was seen to be false.
\end{enumerate}
