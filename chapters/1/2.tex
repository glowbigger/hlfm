\section{Truth functions and truth tables}

\solutions{}

\begin{enumerate}
  \setcounter{enumi}{2}
  \item % 3
    Because truth tables are tedious to write, only (a) has been done. The rest are done similarly.
    \begin{enumerate}[(\alph*)]
      \item
        \begin{tabular}{ccccc}
          \(((\sim\)&
          \(p)\)&
          \(\wedge\)&
          \((\sim\)&
          \(q))\)\\\hline

          F&
          T&
          \underline{F}&
          F&
          T\\

          F&
          T&
          \underline{F}&
          T&
          F\\

          T&
          F&
          \underline{F}&
          F&
          T\\

          T&
          F&
          \underline{T}&
          T&
          F
        \end{tabular}

        % \item
          % \begin{tabular}{ccccccccc}
          %   \(\sim\)&
          %   \(((p\)&
          %   \(\ra\)&
          %   \(q)\)&
          %   \(\ra\)&
          %   \((\sim\)&
          %   \((q\)&
          %   \(\ra\)&
          %   \(p))\)\\\hline
          % \end{tabular}
    \end{enumerate}

  \item % 4
    When \(p\) and \(q\) take on particular values, the values of \(((\sim p) \lor q)\) and \((p \ra q)\) are identical. This can be shown by constructing truth tables, but that process is omitted here. Similarly, \(((\sim p) \ra (q \lor r))\) can be shown to give rise to the same truth function as \(((\sim q) \ra ((\sim r) \ra p))\) by constructing truth tables.

  \item % 5
    The statement forms (a), (b), and (d) are tautologies.

  \item % 6
    Because truth tables are tedious to write, only (a) has been done. The rest are done similarly.
    \begin{enumerate}[(\alph*)]
      \item The truth table for \((p \ra q)\):

        \begin{tabular}{ccc}
          \((p\)&
          \(\ra\)&
          \(q)\)\\\hline

          T&
          \underline{T}&
          T\\

          T&
          \underline{F}&
          F\\

          F&
          \underline{T}&
          T\\

          F&
          \underline{T}&
          F
        \end{tabular}

        The truth table for \(((\sim q) \ra (\sim p))\):

        \begin{tabular}{ccccc}
          \(((\sim\)&
          \(q)\)&
          \(\ra\)&
          \((\sim\)&
          \(p))\)\\\hline

          F&
          T&
          \underline{T}&
          F&
          T\\

          F&
          T&
          \underline{T}&
          T&
          F\\

          T&
          F&
          \underline{F}&
          F&
          T\\

          T&
          F&
          \underline{T}&
          T&
          F
        \end{tabular}

        Notice that when \(p\) and \(q\) take on the same values in both truth tables, the underlined values, which indicate the values of the statement forms, are identical. Therefore, the two statement forms are logically equivalent.
    \end{enumerate}

  \item % 7
    When \(p\) and \(q\) are both true, the value \((((\sim p) \ra q) \ra (p \ra (\sim q)))\) is false, so the statement form is not a tautology.

    Let \(\cua = \cub = (p \ra p)\). Notice that this is a tautology, and for that matter, any tautology can be substituted here. The value of \((((\sim \cua) \ra \cub) \ra (\cua \ra (\sim \cub)))\) can then be seen to be a contradiction, in the same way that the above statement form was seen to be false.
\end{enumerate}
