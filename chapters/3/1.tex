% TODO this should be immediate!
% ~ FA P(x) = not everything has P
% FA ~ P(x) = nothing has P

\section{Predicates and sequences}

\subsection*{A warning}

The reader of \textit{Logic For Mathematicians} of course already has knowledge of formal mathematics regarding quantified statements. There is a common unspoken convention in math that statements are to quantified implicitly if quantifiers are absent. A statement like \(x < y\) might be written to be short for \((\forall x) (\forall y) (x < y)\), where \(x\) and \(y\) are integers.

But this assumption cannot be held while reading Chapter 3 of the textbook or when studying first-order logic in general. Some strings of symbols, which will later be called \wfs{}, are neither true nor false, and the reason for this is the absence of quantifiers. For example, the string \(x < y\) is assumed to be indeterminate in the sense that it is neither true nor false, and likewise \((\forall x) x < y\) is also indeterminate. The truth of these formulas depend on how \(x\) and \(y\) are evaluated (\(x = y = 3\) would yield a false evaluation, for example), which is an idea which will be formalized in the last chapter of this section.

We will see later that all \textit{closed formulas}, which refer to formulas where all variables are quantified, are true or false, as expected. However, not all unquantified formulas are indeterminate. Consider \(x = x\), which is true, despite the formula having no quantifiers.

\solutions{}

\begin{enumerate}
  \item % 1
    \begin{enumerate}[label=(\alph*), align=left]
      \item \(\sim(\forall x)(F(x) \ra D(x))\)
      \item \((\exists x)(F(x) \land C(x) \land (\sim D(x)))\)
      \item \((\exists x)(T(x) \land L(x)) \ra (\forall x)(T(x) \ra L(x))\)
      \item \((\forall x)(E(x) \lor O(x))\)
      \item \(\sim(\exists x)(E(x) \land O(x))\)
      \item \((\exists x)(P(x) \land (\forall y)(P(y) \ra H(x, y)))\)
      \item \((\forall x)(E(x) \ra (\forall y)(M(y) \ra H(x, y)))\)
    \end{enumerate}

  \item % 2
    \begin{enumerate}
      \item \(\sim(\forall x)(C(x) \ra T(x))\)

            \((\exists x)(C(x) \land \sim T(x))\)

      \item \(\sim(\forall x)(P(x) \ra (\sim L(x) \land \sim S(x)))\)

            \((\exists x)(P(x) \land (L(x) \lor S(x)))\)

      \item \((\forall x)(\forall y)((M(x) \land E(y)) \ra \sim H(x, y))\)

            \(\sim(\exists x)(\exists y)(M(x) \land E(y) \land H(x, y))\)

      \item \((\forall x)(N(x) \lor S(x))\)

            \(\sim (\exists x)((\sim N(x)) \land (\sim S(x)))\)
    \end{enumerate}

  \item % 3
    % TODO do this later when you are sure of the answer
\end{enumerate}
