\section{First order languages}

Here we will define a mathematical structure which is not a formal system as it has no rules or deductions or axioms. It does have an alphabet, which is specified below. In fact, it may be thought of, structurally speaking, as only an alphabet, a set of symbols.

A \textit{first order language} \(\cul\) will have as its alphabet of symbols:
\begin{itemize}
  \item the countably infinite list of \textit{variables} \(x_1, x_2, \dots\)
  \item some or none of the \textit{constants} \(a_1, a_2, \dots\)
  \item some or none of the \textit{predicate letters} \(A^1_1, A^1_2, \dots; A^2_1, A^2_2, \dots; A^3_1, A^3_1, \dots; \dots\)
  \item some or none of the \textit{function letters} \(f^1_1, f^1_2, \dots; f^2_1, f^2_2, \dots; f^3_1, f^3_1, \dots; \dots\)
    \begin{itemize}
      \item the predicate and function letters are countably infinite lists of countably infinite lists
      \item the subscripts are used to distinguish different functions of the same arity, while the superscript indicates the arity
      \item the function letters are not strictly necessary, as they can be expressed as relations as well, but the redundancy is kept for purposes of intuitive clarity
    \end{itemize}
  \item the left and right parentheses ( and ) and the comma , as \textit{punctuation symbols}
  \item the \textit{connectives} \(\sim\) and \(\ra\)
  \item the quantifier \(\forall\)
    \begin{itemize}
      \item the existential quantifier \(\exists\) can be expressed in terms of the universal quantifier along with the \(\sim\) connective, so it is not included
    \end{itemize}
\end{itemize}

\note{} Two first order languages differ only in the symbols which are in the alphabet, particularly which constants, predicate letters and function letters are included. For example, one first order language might only include \(a_1\) and \(f^1_1\), and another might only include \(A^1_1\).

The alphabet will be part of a formal system described in the next chapter which will make clear the logical rules in a first-order system. Since no rules of deduction or axioms have been specified, studying first order languages, which is the subject of this chapter, will be limited to studying not the valid rules of transformations of strings of symbols that correspond to valid deductions, but rather what the valid strings are and how they are to be interpreted.

\setcounter{definition}{5}
\begin{definition}
  A \textit{term} in a first order language \cl{} is defined as follows
  \begin{enumerate}[(i)]
    \item Variables and individual constants are terms.
    \item If \(f^n_i\) is a function letter in \cl{}, and \(t_1, \dots, t_2\) are terms in \cl{}, then \(f^n_i(t_1, \dots, t_2)\) is a term in \cl{}.
    \item The set of all terms is generated as in (i) and (ii).
  \end{enumerate}

  An \textit{atomic formula} in \cl{} is defined by: if \(A^k_j\) is a predicate letter in \cl{} and \(t_1, \dots, t_n\) are terms in \(L\), then \(A^k_j(t_1, \dots, t_n)\) is an atomic formula of \cl{}.

  A \textit{well-formed formula} of \cl{} is defined by:
  \begin{enumerate}[(i)]
    \item Every atomic formula of \cl{} is a \wf{} of \cl{}.
    \item If \(\cua\) and \(\cub\) are \wfs{}, so are \((\sim\cua), (\cua \ra \cub)\) and \((\forall x_i)\cua\), where \(x_i\) is any variable.
    \item The set of all \wfs{} of \cl{} is generated as in (i) and (ii)
  \end{enumerate}

  Terms are to be considered as objects in the language, a \wf{} as a statement, and an atomic formula as the most simple kind of statement.
\end{definition}

\medskip

The symbols \(\exists, \land\) and \(\lor\) are treated as shorthand.
\begin{itemize}
  \item \((\exists x_i) \cua\) is an abbreviation for \((\sim((\forall x_i)(\sim\cua))\)
  \item \((\cua \land \cub)\) is an abbreviation for \((\sim(\cua \ra (\sim\cub)))\)
  \item \((\cua \lor \cub)\) is an abbreviation for \(((\sim\cua) \ra \cub)\)
\end{itemize}

\note{} Unlike in chapter 2, this manual will follow the book's style for omitting parentheses. That is, a \(\sim\) will be presumed to apply to the shortest possible subsequent \wf{}. Also notice that this is how the \(\forall\) quantifier is treated, by definition.

\setcounter{definition}{7}
\begin{definition}
  In the \wf{} \((\forall x_i)\cua\), we say that \(\cua\) is the \textit{scope} of the quantifier. When \((\forall x_i)\cua\) occurs as a subformula of a \wf{} \(\cub\), the scope of the quantifier \((\forall x_i)\) is said to be \(\cua\) in \(\cub\).

  A variable \(x_i\) in a \wf{} is said to be \textit{bound} if it occurs within the scope of a \((\forall x_i)\) in the \wf{} or if it is the \(x_i\) in a \((\forall x_i)\). A variable which is not bound is said to be \textit{free}.

  \note{} One point of confusion is that the meaning of a formula \((\forall x_i) \cua\) is reliant on its \textit{free} variables, not its bound ones, which might be contrary to one's intuitions about the words \textit{free} and \textit{bound}.
\end{definition}

From here on out, if \(\cua(x_i)\) is a \wf{} in which \(x_i\) occurs free, then \(\cua(t)\) will refer to \(\cua(x_i)\) with all \textit{free} occurrences of \(x_i\) replaced with \(t\). So if \(\cua(x_i)\) is \((\forall x_2) A^1_1(x_i) \ra (\forall x_i) A^1_1(x_i))\), then \(\cua(t)\) is \((\forall x_2) A^1_1(t) \ra (\forall x_i) A^1_1(x_i))\). We will want to only substitute \(t\) for \(x_i\) if it does not interact with quantifiers in \(\cua(x_i)\). In the previous example, \(x_2\) would be a different substitution that any other variable. For this reason, we need the next definition, which is as important as it is confusing. A few equivalent definitions will be provided.

\setcounter{definition}{10}
\begin{definition}
  Let \(\cua\) be any \wf{} of \cl{}. A term \(t\) is \textit{free for \(x_i\) in \(\cua\)} if \(x_i\) does not occur free in \(\cua\) within the scope of a \((\forall x_j)\), where \(x_j\) is any variable occurring in \(t\).

  Equivalently, a term \(t\) is \textit{free for \(x_i\) in \(\cua\)} if substituting \(t\) for any free instance of \(x_i\) in \(\cua\) would not introduce any new bound variables.

  Equivalently, a term \(t\) is \textit{free for \(x_i\) in \(\cua\)} if any variable in \(t\) is free in \(\cua\) after substituting it for any free instance of \(x_i\).

  An algorithm for determining whether a term \(t\) is \textit{free for \(x_i\) in \(\cua\)} goes as follows:

  \begin{enumerate}
    \item Find all free instances of \(x_i\) in \(\cua\).
    \item For each free instance \(x_j\), repeat the following step:
      \begin{enumerate}
        \item For each variable \(x_k\) in \(t\), repeat the following steps:
        \begin{enumerate}
          \item Substitute \(x_k\) for \(x_j\).
          \item If \(x_k\) is bound in \(\cua\), then \(t\) is not free for \(x_i\), and terminate the algorithm.
        \end{enumerate}
      \end{enumerate}
    \item Conclude that \(t\) is free for \(x_i\).
  \end{enumerate}

  \note{} A term being free for a variable \(x_i\) does not necessarily indicate that it may be substituted for that variable, because the variable \(x_i\) may be bound. But if \(x_i\) only occurs free, then a term being free for it is equivalent to a term being substitutable for it. Therefore ``\(t\) being free for \(x_i\)'' can be thought of as ``\(t\) being substitutable for free instances of \(x_i\).''

  \note{} It is easy to confirm that if \(x_i\) occurs only bound in \(\cua\), then any term is free for it. Also, for any \wf{} and any variable \(x_i\), \(x_i\) is free for itself in \(\cua\).
\end{definition}

\solutions{}

\begin{enumerate}
  \setcounter{enumi}{3}
  \item % 4
    The set of terms in a first order language with no function letters is just the set of the variables and the individual constants.

  \item % 5
    The set of terms is \(f^1_1(x_1), f^1_1(x_2), f^1_1(x_3), \dots\).

  \item % 6
    The formulas that are well-formed formulas are (a), (d), (e), (g), and (h).

    \note{} The answer key in the book omits (d) as a well-formed formula, but this appears to be wrong.

  \item % 7
    \begin{enumerate}[label = (\alph*), align = left]
      \item free
      \item bound, bound
      \item bound, bound, free
      \item free, free, free, free
    \end{enumerate}

    Since all occurrences of \(x_2\) in the \wfs{} are bound, the term \(f^2_1(x_1, x_3)\) (and any other term) is free for \(x_2\) in the \wfs{}.

  \item % 8
    Suppose that \(x_j\) is free for \(x_i\) in \(\cua(x_i)\). Then \(x_i\) does not occur free in the scope of a \((\forall x_j)\) in \(\cua(x_i)\). The goal is to show that \(x_j\) does not occur free in the scope of a \((\forall x_i)\) in \(\cua(x_j)\).

    Now consider an occurrence of an \(x_j\) in \(\cua(x_j)\), which is either (1) a substitution for an \(x_i\) in \(\cua(x_i)\) or (2) not a substitution for an \(x_i\) in \(\cua(x_i)\). If (1), then since \(x_1\) was assumed to occur free in \(\cua(x_i)\), it must not be in the scope of a \((\forall x_i)\), and so the substituted \(x_j\) must also not be in the scope of a \((\forall x_i)\). If (2), then it must occur the same as it does in \(\cua(x_i)\), a \wf{} in which it is assumed to occur bound, and therefore it does not occur free in the scope of a \((\forall x_i)\), or any quantifier for that matter.

    Since in both (1) and (2), \(x_j\) does not occur free in the scope of a \((\forall x_i)\) in \(\cua(x_j)\), it can be concluded that \(x_i\) is free for \(x_j\) in \(\cua(x_j)\).

    \note{} The hypothesis that \(x_j\) is free for \(x_i\) is never used in the proof. It seems that there is either a mistake in the proof or in the exercise. The hint/proof in the back of the book seems to agree with the proof given here.

  \item % 9
    \begin{enumerate}[label = (\alph*), align = left]
      \item Since \((\forall x_1)\) and \((\forall x_3)\) never occur, \underline{\(t\) is free for \(x_1\)}.
      \item Since \(x_1\) only occurs bound, \underline{\(t\) is free for \(x_1\)}.
      \item Since \(x_1\) occurs free in the scope of a \((\forall x_3)\), \underline{\(t\) is not free for \(x_1\)}.
      \item Since \(x_1\) occurs free in the scope of a \((\forall x_3)\), \underline{\(t\) is not free for \(x_1\)}.
    \end{enumerate}

  \item % 10
    Here, the (a), (b), (c), and (d) refer to the \wfs{} in the exercise 9 and not the terms in exercise 10.

    \note{} There is a mistake in the text, the exercise should read ``Repeat Exercise 9...'' instead of ``Repeat Exercise 6...''.

    Let \(t = x_2\).
    \begin{enumerate}[label = (\alph*), align = left]
      \item Since \(x_1\) occurs free in the scope of a \((\forall x_2)\), \underline{\(t\) is not free for \(x_1\)}.
      \item Since \(x_1\) only occurs bound, \underline{\(t\) is free for \(x_1\)}.
      \item Since \(x_1\) does not occur in the scope of a \((\forall x_2)\), it does not occur free in it, so \underline{\(t\) is free for \(x_1\)}.
      \item Since \(x_1\) occurs free in the scope of a \((\forall x_2)\), \underline{\(t\) is not free for \(x_1\)}.
    \end{enumerate}

    Let \(t = x_3\).
    \begin{enumerate}[label = (\alph*), align = left]
      \item Since a \((\forall x_3)\) never occurs, \underline{\(t\) is free for \(x_1\)}.
      \item Since \(x_1\) only occurs bound, \underline{\(t\) is free for \(x_1\)}.
      \item Since \(x_1\) occurs free in the scope of a \((\forall x_3)\), \underline{\(t\) is not free for \(x_1\)}.
      \item Since \(x_1\) occurs free in the scope of a \((\forall x_3)\), \underline{\(t\) is not free for \(x_1\)}.
    \end{enumerate}

    Let \(t = f^2_1(a_1, x_1)\).
    \begin{enumerate}[label = (\alph*), align = left]
      \item Since a \((\forall x_1)\) never occurs, \underline{\(t\) is free for \(x_1\)}.
      \item Since \(x_1\) only occurs bound, \underline{\(t\) is free for \(x_1\)}.
      \item Since a \((\forall x_1)\) never occurs, \underline{\(t\) is free for \(x_1\)}.
      \item Since a \((\forall x_1)\) never occurs, \underline{\(t\) is free for \(x_1\)}.
    \end{enumerate}

    Let \(f^3_1(x_1, x_2, x_3)\).
    \begin{enumerate}[label = (\alph*), align = left]
      \item Since \(x_1\) occurs free in the scope of a \((\forall x_2)\), \underline{\(t\) is not free for \(x_1\)}.
      \item Since \(x_1\) only occurs bound, \underline{\(t\) is free for \(x_1\)}.
      \item Since \(x_1\) occurs free in the scope of a \((\forall x_3)\), \underline{\(t\) is not free for \(x_1\)}.
      \item Since \(x_1\) occurs free in the scope of a \((\forall x_2)\) and a \((\forall x_3)\), \underline{\(t\) is not free for \(x_1\)}.
    \end{enumerate}
\end{enumerate}
