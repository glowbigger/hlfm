\section{Consistency and models}

There may be a demonstration of showing that one system is consistent as long as another one is consistent. This is called \textit{relative consistency}. For example, suppose that an extension of a system \(S\) is consistent. Then \(S\) must be consistent by the following proposition.

\setcounter{definition}{10}
\begin{proposition}
  Let \(S^*\) be an extension of \(S\). If \(S^*\) is consistent, then \(S\) is consistent.

  \begin{proof}
    By contrapositive. Suppose that \(S\) is not consistent. Then there exists a proof of \(\cua\) and \(\sim\cua\) in \(S^*\). A proof in a system is a proof in any extension of the system, by definition of an extension, so there are proofs of \(\cua\) and \(\sim\cua\) in \(S^*\), so \(S^*\) is inconsistent.
  \end{proof}
\end{proposition}
